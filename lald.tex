\documentclass{article}
\usepackage{amsmath}
\usepackage{amssymb}
\usepackage{amsthm}
\usepackage{amssymb}
\usepackage{mathdots}
\usepackage[pdftex]{graphicx}
\usepackage{fancyhdr}
\usepackage[margin=1in]{geometry}
\usepackage{multicol}
\usepackage{bm}
\usepackage{listings}
\PassOptionsToPackage{usenames,dvipsnames}{color}  %% Allow color names
\usepackage{pdfpages}
\usepackage{algpseudocode}
\usepackage{tikz}
\usepackage{enumitem}
\usepackage[T1]{fontenc}
\usepackage{inconsolata}
\usepackage{framed}
\usepackage{wasysym}
\usepackage[thinlines]{easytable}
\usepackage{hyperref}
\hypersetup{
    colorlinks=true,
    linkcolor=blue,
    filecolor=magenta,      
    urlcolor=blue,
}

\newcommand{\abs}[1]{\lvert #1 \rvert}
\newcommand{\absfit}[1]{\left\lvert #1 \right\rvert}
\newcommand{\norm}[1]{\left\lVert #1 \right\rVert}
\newcommand{\eval}[3]{\left[#1\right]_{#2}^{#3}}
\renewcommand{\(}{\left(}
\renewcommand{\)}{\right)}
\newcommand{\floor}[1]{\left\lfloor#1\right\rfloor}
\newcommand{\ceil}[1]{\left\lceil#1\right\rceil}
\newcommand{\pd}[1]{\frac{\partial}{\partial #1}}
\newcommand{\inner}[1]{\langle#1\rangle}
\newcommand{\cond}{\bigg|}
\newcommand{\rank}[1]{\mathbf{rank}(#1)}
\newcommand{\range}[1]{\mathbf{range}(#1)}
\newcommand{\nullsp}[1]{\mathbf{null}(#1)}
\newcommand{\repr}[1]{\left\langle#1\right\rangle}

\DeclareMathOperator{\Var}{Var}
\DeclareMathOperator{\tr}{tr}
\DeclareMathOperator{\Tr}{\mathbf{Tr}}
\DeclareMathOperator{\diag}{\mathbf{diag}}
\DeclareMathOperator{\dist}{\mathbf{dist}}
\DeclareMathOperator{\prob}{\mathbf{prob}}
\DeclareMathOperator{\dom}{\mathbf{dom}}
\DeclareMathOperator{\E}{\mathbf{E}}
\DeclareMathOperator{\R}{\mathbb{R}}
\DeclareMathOperator{\var}{\mathbf{var}}
\DeclareMathOperator{\quartile}{\mathbf{quartile}}
\DeclareMathOperator{\conv}{\mathbf{conv}}
\DeclareMathOperator{\VC}{VC}
\DeclareMathOperator*{\argmax}{arg\,max}
\DeclareMathOperator*{\argmin}{arg\,min}
\DeclareMathOperator{\Ber}{Bernoulli}
\DeclareMathOperator{\NP}{\mathbf{NP}}
\DeclareMathOperator{\coNP}{\mathbf{coNP}}
\DeclareMathOperator{\TIME}{\mathsf{TIME}}
\DeclareMathOperator{\polytime}{\mathbf{P}}
\DeclareMathOperator{\PH}{\mathbf{PH}}
\DeclareMathOperator{\SIZE}{\mathbf{SIZE}}
\DeclareMathOperator{\ATIME}{\mathbf{ATIME}}
\DeclareMathOperator{\SPACE}{\mathbf{SPACE}}
\DeclareMathOperator{\ASPACE}{\mathbf{ASPACE}}
\DeclareMathOperator{\NSPACE}{\mathbf{NSPACE}}
\DeclareMathOperator{\Z}{\mathbb{Z}}
\DeclareMathOperator{\N}{\mathbb{N}}
\DeclareMathOperator{\EXP}{\mathbf{EXP}}
\DeclareMathOperator{\NEXP}{\mathbf{NEXP}}
\DeclareMathOperator{\NTIME}{\mathbf{NTIME}}
\DeclareMathOperator{\DTIME}{\mathbf{DTIME}}
\DeclareMathOperator{\poly}{poly}
\DeclareMathOperator{\BPP}{\mathbf{BPP}}
\DeclareMathOperator{\ZPP}{\mathbf{ZPP}}
\DeclareMathOperator{\RP}{\mathbf{RP}}
\DeclareMathOperator{\coRP}{\mathbf{coRP}}
\DeclareMathOperator{\BPL}{\mathbf{BPL}}
\DeclareMathOperator{\IP}{\mathbf{IP}}
\DeclareMathOperator{\PSPACE}{\mathbf{PSPACE}}
\DeclareMathOperator{\NPSPACE}{\mathbf{NPSPACE}}
\DeclareMathOperator{\SAT}{\mathsf{SAT}}
\DeclareMathOperator{\NL}{\mathbf{NL}}
\DeclareMathOperator{\PCP}{\mathbf{PCP}}
\DeclareMathOperator{\PP}{\mathbf{PP}}
\DeclareMathOperator{\cost}{cost}
\let\Pr\relax
\DeclareMathOperator*{\Pr}{\mathbf{Pr}}

\definecolor{shadecolor}{gray}{0.95}

\theoremstyle{plain}
\newtheorem*{lem}{Lemma}

\theoremstyle{plain}
\newtheorem*{claim}{Claim}

\theoremstyle{definition}
\newtheorem*{answer}{Answer}

\newtheorem{theorem}{Theorem}[section]
\newtheorem*{thm}{Theorem}
\newtheorem{corollary}{Corollary}[theorem]
\newtheorem{lemma}[theorem]{Lemma}

\renewcommand{\headrulewidth}{0.4pt}
\renewcommand{\footrulewidth}{0.4pt}

\setlength{\parindent}{0pt}
\setcounter{secnumdepth}{0}

\pagestyle{fancy}

\renewcommand{\thefootnote}{\fnsymbol{footnote}}

\title{Linear Algebra and Learning from Data Problem Sets}

\begin{document}

\maketitle

\section{Problem Set I.3}
\begin{enumerate}[label*=\arabic*.,ref=\arabic*]
\item Problem 5: Find four matrices $A_1$ to $A_4$.
\begin{shaded}
\begin{align*}
A_2 = \begin{bmatrix} 1 & 3 & 1 \\ 0 & 1 & 0 \end{bmatrix} \\
r = m < n, m = 2, n = 6, r = 2 \\
A_3 = \begin{bmatrix} 1 & 3 \\ 2 & 1 \\ 3 & 2 \end{bmatrix} \\
r = n < m, m = 3, n = 2, r = 2 \\
A_4 = \begin{bmatrix} 0 & 1 \\ 0 & 0 \end{bmatrix} \\
r < m, r < n, m = 2, n = 2, r = 1
\end{align*}
\end{shaded}
\end{enumerate}

\section{Problem Set I.4}
\begin{enumerate}[label*=\arabic*.,ref=\arabic*]
\item Problem 3: What lower triangular matrix E puts A into upper triangular form $\textit{EA = U}$? Multiply by $E^{-1} = L$ to factor $A$ into $LU$: $A = \begin{bmatrix} 2 & 1 & 0 \\ 0 & 4 & 2 \\ 6 & 3 & 5 \end{bmatrix}$
\begin{shaded}
$\begin{bmatrix} 1 & 0 & 0 \\ 0 & 0 & 1 \\ 0 & 1 & 0 \end{bmatrix} \begin{bmatrix} 2 & 1 & 0 \\ 0 & 4 & 2 \\ 6 & 3 & 5 \end{bmatrix} = \begin{bmatrix} 2 & 1 & 0 \\ 6 & 3 & 5 \\ 0 & 4 & 2 \end{bmatrix}$ \\

$\begin{bmatrix} 1 & 0 & 0 \\ -3 & 1 & 0 \\ 0 & 0 & 1 \end{bmatrix} \begin{bmatrix} 2 & 1 & 0 \\ 6 & 3 & 5 \\ 0 & 4 & 2 \end{bmatrix} = \begin{bmatrix} 2 & 1 & 0 \\ 0 & 0 & 5 \\ 0 & 4 & 2 \end{bmatrix}$ \\

$\begin{bmatrix} 1 & 0 & 0 \\ 0 & 0 & 1 \\ 0 & 1 & 0 \end{bmatrix} \begin{bmatrix} 2 & 1 & 0 \\ 0 & 0 & 5 \\ 0 & 4 & 2 \end{bmatrix} = \begin{bmatrix} 2 & 1 & 0 \\ 0 & 4 & 2 \\ 0 & 0 & 5 \end{bmatrix}$ \\

$P_2E_1P_1A$ \\

$\begin{bmatrix} 1 & 0 & 0 \\ -3 & 1 & 0 \\ 0 & 0 & 1 \end{bmatrix} \begin{bmatrix} 1 & 0 & 0 \\ 0 & 0 & 1 \\ 0 & 1 & 0 \end{bmatrix} = \begin{bmatrix} 1 & 0 & 0 \\ -3 & 0 & 1 \\ 0 & 1 & 0 \end{bmatrix}$ \\

$E_1P_1$ \\

$\begin{bmatrix} 1 & 0 & 0 \\ 0 & 0 & 1 \\ 0 & 1 & 0 \end{bmatrix} \begin{bmatrix} 1 & 0 & 0 \\ -3 & 0 & 1 \\ 0 & 1 & 0 \end{bmatrix} = \begin{bmatrix} 1 & 0 & 0 \\ 0 & 1 & 0 \\ -3 & 0 & 1 \end{bmatrix}$ \\

$P_2E_1P_1 = E$ \\

$\begin{bmatrix} 1 & 0 & 0 \\ 0 & 1 & 0 \\ 3 & 0 & 1 \end{bmatrix} \begin{bmatrix} 2 & 1 & 0 \\ 0 & 4 & 2 \\ 6 & 3 & 5 \end{bmatrix} = \begin{bmatrix} 2 & 1 & 0 \\ 0 & 4 & 2 \\ 0 & 0 & 5 \end{bmatrix}$ \\

$EA = U$ \\

$\begin{bmatrix} 1 & 0 & 0 \\ 0 & 1 & 0 \\ 3 & 0 & 1 \end{bmatrix} \begin{bmatrix} 2 & 1 & 0 \\ 0 & 4 & 2 \\ 0 & 0 & 5 \end{bmatrix} = \begin{bmatrix} 2 & 1 & 0 \\ 0 & 4 & 2 \\ 6 & 3 & 5 \end{bmatrix}$ \\

$E^{-1} = L, LU = A$
\end{shaded}

\item Problem 4
\begin{shaded}
\begin{enumerate}[label=\alph*)]
\item 
$A = L, EA = I, E = L^{-1}$ \\
$L^{-1}L = I$
\item 
$E^{-1} = L$ \\
$E^{-1}EA = E^{-1}I$ \\
$E^{-1}E = I$ \\
$IA = E^{-1}I$ \\
$A = E^{-1} = L$ 
\end{enumerate}
\end{shaded}

\item Problem 6
\begin{shaded}
c = 2 gives zero in second pivot, c = 1 gives zero in third pivot \\

$\begin{bmatrix} 1 & c & 0 \\ 0 & 4-2c & 1 \\ 0 & 5-3c & 1 \end{bmatrix}$ \\

$5 - 3c = 4 - 2c$ \\

$\begin{bmatrix} 1 & 1 & 0 \\ 2 & 4 & 1 \\ 3 & 5 & 1 \end{bmatrix} \rightarrow \begin{bmatrix} 1 & 1 & 0 \\ 0 & 2 & 1 \\ 0 & 2 & 1 \end{bmatrix}$
\end{shaded}

\item Problem 8: Triadiagonal matrices have zero entries except on the main diagonal and the two adjacent diagonals. Factors these into $A = LU$. $A = \begin{bmatrix} 1 & 1 & 0 \\ 1 & 2 & 1 \\ 0 & 1 & 2 \end{bmatrix}$ and $A = \begin{bmatrix} a & a & 0 \\ a & a+b & b \\ 0 & b & b+c \end{bmatrix}$
\begin{shaded}

$L = \begin{bmatrix} 1 & 0 & 0 \\ 1 & 1 & 0 \\ 0 & 1 & 1 \end{bmatrix} U = \begin{bmatrix} 1 & 1 & 0 \\ 0 & 1 & 1 \\ 0 & 0 & 1 \end{bmatrix}$ \\

$L = \begin{bmatrix} 1 & 0 & 0 \\ 1 & 1 & 0 \\ 0 & 1 & 1 \end{bmatrix} U = \begin{bmatrix} a & a & 0 \\ 0 & b & b \\ 0 & 0 & c \end{bmatrix}$
\end{shaded}

\item Problem 9: If $A$ has pivots 5, 9, 3 with no row exchanges, what are the pivots for the upper left 2 by 2 submatrix $A_2$ (without row 3 and column 3)?
\begin{shaded}
$A = \begin{bmatrix} 5 & a & b \\ 0 & 9 & c \\ 0 & 0 & 3 \end{bmatrix} A_2 = \begin{bmatrix} 5 & a \\ 0 & 9 \end{bmatrix}$ \\

$A_2$ pivots are 5 and 9
\end{shaded}

\item Problem 11: Question for $A = \begin{bmatrix} 1 & 3 \\ 2 & 4\end{bmatrix}$: Apply complete pivoting to produce $P_{1}AP_{2} = LU$.
\begin{shaded}
$\begin{bmatrix} \frac{3}{4} & 1 \\ 1 & 0 \end{bmatrix} \begin{bmatrix} 2 & 4 \\ -\frac{1}{2} & 0 \end{bmatrix} = \begin{bmatrix} 1 & 3 \\ 2 & 4 \end{bmatrix}$ \\

$\begin{bmatrix} 1 & 0 \\ \frac{3}{4} & 1 \end{bmatrix} \begin{bmatrix} 4 & 2 \\ 0 & -\frac{1}{2} \end{bmatrix} = \begin{bmatrix} 4 & 2 \\ 3 & 1 \end{bmatrix}$
\end{shaded}

\item Problem 12: If the short wide matrix $A$ has $m < n$, how does elimination show that there are nonzero solutions to $Ax = 0$?  
\begin{shaded}
$A = \begin{bmatrix} 1 & 2 & 3 \\ 2 & 5 & 6 \end{bmatrix}$ \\

$L = \begin{bmatrix} 1 & 0 \\ 2 & 1 \end{bmatrix} U = \begin{bmatrix} 1 & 2 & 3 \\ 0 & 1 & 0 \end{bmatrix}$
\end{shaded}
\end{enumerate}


\section{Problem Set I.5}
\section{Problem Set I.6}
\section{Problem Set I.7}
\section{Problem Set I.8}
\section{Problem Set I.9}
\section{Problem Set I.10}
\section{Problem Set I.11}
\section{Problem Set I.12}
\section{Problem Set II.1}
\section{Problem Set II.2}
\section{Problem Set II.3}
\section{Problem Set II.4}

\end{document}
